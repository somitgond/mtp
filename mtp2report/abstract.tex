\begin{flushright}
\huge{\section*{Abstract}}
\end{flushright}
\addcontentsline{toc}{section}{Abstract}
Today's world can't be imagined without internet. Router form the backbone of internet, they transfer packets(data) from one link to another so, internet can be considered as huge web of interconnected routers. Everyday internet traffic keeps rising, but internet's infrastructure doesn't scale as fast. This creates huge load on existing infrastructure, leading to scenarios of network congestion. A Network congestion forms at a router when rate of incoming packets is greater than bandwidth of outgoing link. Various methods have been developed over the years to combat network congestion. Routers have finite buffer and they drop packet when it gets full. Different flavours of TCP have been developed to efficiently handle congestion. TCP takes multiple packet loss as a sign of network congestion so it starts to slow down its sending rates. Routers leverage this behaviour of TCP in their favour they can drop packets early to notify clients that congestion is happening or will happen. Router can do this by managing its queue size. Threshold based queueing policy is one of the methods where router starts to drop packets when queue size exceeds a threshold. Global synchronization is one of the issues arising due to this policy in which either all clients back off simultaneously or start sending packets simultaneously, resulting in underutilization of bandwidth of link. It is essential to set threshold such that global synchronization doesn't happen. We want to use data driven tecniques to compute the queue threshold ($ q_{th} $). 
